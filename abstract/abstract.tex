\section*{Abstract}

This paper compares two line-following robots built from the LEGO Mindstorms kit and programmed with ROS2 to solve a simply connected maze. The robots were compared using three metrics: time, energy consumption, and reliability. Both robots used a single colour sensor and differential drive to scan and traverse the maze. The only difference between the two robots was that one could rotate the sensor 180 degrees to scan intersections while the other could not. Both robots drove eleven recorded test runs through the maze to analyse which robot performed better on the three metrics. The robots followed the lines accurately and correctly navigated all intersections. However, due to measurement inaccuracies, it was not possible to determine which robot was more energy-efficient. Nevertheless, the analysis showed that the robot with the rotatable sensor is generally faster at scanning and manoeuvring intersections. Based on these findings, this paper concludes that using a 180-degree rotatable colour sensor can improve scanning and navigation in a maze.