\section{Introduction} \label{sec:introduction}

Line-following robots are low-cost and easy-to-build mobile units that have gained increasing attention in recent years. Applications for such robots range from in-house transportation to path guidance in museums or shopping malls. For these robots to function efficiently, many design decisions must be made correctly regarding sensors, propulsion, and software configurations. Researchers conducted several studies to understand how different configurations affect the robot's efficiency in line-following \cite{baharuddin, pakdaman, almeida}. While most studies examine robot designs for following continuous lines, there needs to be more research on robot configurations for following lines with intersections that require scanning and navigation. This study addresses this research gap by comparing two robot configurations in maze solving. The compared robots have a single colour sensor mounted in front of the robot and use the same drive system. The difference between the robots is that one can rotate the sensor horizontally 180 degrees using a motor, while the other robot's sensor is fixed.

This paper first shows and discusses the relevant literature in Chapter \ref{sec:litrev}. Then, it explains the robot configurations and the methodology used to test them in Chapter \ref{sec:methodology}. Next, the thesis discusses the robot-specific test results in Chapter \ref{sec:results}. 
Chapter \ref{sec:discussion} compares results and answers the research question. Chapter \ref{sec:conclusion} briefly explains the relevance of the results to other robots and makes suggestions for future work.
