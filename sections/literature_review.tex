\section{Literature review} \label{sec:litrev}

Line-following robots have been a topic of interest in robotics for many years. The goal is to enable mobile robots to navigate complex environments by following a line with high accuracy and reliability \cite{pakdaman}. Various studies explore different approaches for designing robots and implementing the algorithms. This literature review will analyse the existing research on line-follower robots focusing on the different robot and algorithm configurations. The review will examine the motor and sensor configurations before moving on to the various line-tracking algorithms proposed in the literature.

\subsection{Driving configuration}

This section will discuss the existing line-following robot research focusing on the different approaches to making the robots drive. Only the most relevant approaches will be addressed: Ackerman (car-like) drive, two-wheel differential drive and two-wheel balancing drive. The Ackerman drive is similar to a car's steering mechanism and usually  consists of two fixed wheels at the back of the vehicle and two steerable wheels at the front. The "two-wheel differential drive" and "two-wheel balancing drive" drive systems both consist of two wheels driven by two individual motors that can be controlled individually. The difference is that the differential drive usually uses a third, non-powered, freely rotating wheel, while the balancing configuration uses sensors and algorithms for balancing.

The Stanford cart described in the work of Moravec \cite{moravec} is the second mobile robot after "Shakey the robot" \cite{shakey} and the first line-following robot. The robot used four wheels and Ackerman (car-like) drive for steering. Compared to modern line-follower robots, Ackerman drive is a relatively rare form of locomotion for this type of robot as it is not highly manoeuvrable like other approaches mentioned below.

The most popular approach for line-followers is the two-wheel differential drive with a castor wheel. The authors Pakdaman and Sanaatiyan mention two types of drive: wheels and tank systems \cite{pakdaman}. They state that wheel drive is better for these robots but do not explain why. The robot design used for their study consists of two driven wheels at the back of the robot and a passive castor wheel at the front. This configuration has the advantage of being able to perform tight manoeuvres, turning on the spot and requiring only two motors.

Compared to the stanford cart \cite{moravec}, and the work of Pakdaman and Sanaatiyan \cite{pakdaman}, a more minimalistic robot locomotion design is the mobile inverted pendulum. The concepts got first implemented in the year 1996 \cite{yun-su} and consist of two wheels on which the robot balances. Fifteen years later, a group of researchers \cite{ghani} further developed the technology to allow a self-balancing robot to follow lines. The benefits of this driving configuration are that it has a smaller footprint than three or four-wheeled robots while still being highly manoeuvrable and able to turn on the spot \cite{coelho}.

\subsection{Sensor configurations}
The sensor setup used by a line-following robot is the second part that can significantly influence its accuracy. Different configurations of sensors are used depending on the requirements of the robot. These configurations can differ in the sensors' number, type and positioning. This literature review focuses only on the approaches for sensing visual lines and not other types like magnetic lines.

The work of Pakdaman and Sanaatiyan describes the most popular sensor type used for line-following robots: infrared (IR) sensors \cite{pakdaman}. IR sensors consist of two diodes, one of which sends infrared rays, and the other measures how much of that ray is reflected. To ensure high accuracy, the authors recommend shielding the sensors from ambient light \cite{pakdaman}. The robot designed by Pakdaman and Sanaatiya uses 8 IR sensors in the front of the robot.

Baharuddin et al.  specifically compared different IR sensor configurations, which vary in sensor alignment and the number of sensors \cite{baharuddin}. The authors found that the number of sensors is vital in determining the resolution with which the robot senses the line. They found that at least two IR sensors are required to follow lines efficiently. To also detect intersections, they concluded that four sensors are needed. 

Another, even more, important criterion for good line-following, according to Baharuddin et al., is the positioning of the sensors \cite{baharuddin}. In the configurations with two or four sensors, the sensors are aligned in a line perpendicular to the line the robot needs to follow. The same authors explain why having the sensors too near or too far apart can be problematic and give a formula for the optimal positioning of the sensors depending on the line width \cite{baharuddin}.

Finally, another study explored using a camera to detect the line \cite{dupuis}. The researchers were able to make this work in a simulated environment but failed to transfer it to the physical robot.

\subsection{Line tracking algorithms}

The third and last crucial part influencing the accuracy of the line-following robots is the algorithm used to control the motors based on the sensor input. Researchers have developed many different algorithms, but it is difficult to determine which performs best as it depends on various factors. This section will discuss the most relevant control algorithms for line-following robots: Simple logic, Proportional-Integral-Derivative (PID) controller, and Fuzzy logic, and mention some less common but valid approaches like Neural Networks and Q-learning.

Pakdaman and Sanaatiyan used simple logic to make the robot drive on the line \cite{pakdaman}. They used an if statement to check the values from the IR sensors and turn the wheel motors on or off depending on the values. With this control strategy, the authors concluded that their robot could follow any curves and cycles \cite{pakdaman}.

According to multiple studies, PID controllers are a common approach for line-following robots \cite{gomes, ghani, binugroho, engin}. M. Engin and D. Engin  implemented a PID controller and compared it to a simple “on/off” control \cite{engin}. Their experimental results show that the robot with the PID controller outperforms the simple control in terms of time to complete the track, driving speed, line tracking smoothness and tendency to stray from the line. However, for PID controllers to work well, they need to be tuned, as mentioned by M. Engin and D. Engin \cite{engin}. The tuning can be difficult as it usually has to be done manually.

Fuzzy logic controllers are another control strategy that researchers have found to be efficient for line-following \cite{antonelli, azlan, ibrahim, chen}. These controllers function based on the mathematical system of fuzzy sets developed by Zadeh \cite{zadehfuzzy}. Azlan et al. implemented a fuzzy logic controller and showed its supremacy over the simple logic controllers mentioned earlier \cite{azlan}. They found that fuzzy logic controllers make the robot follow lines faster, more smoothly and more accurately.

Saadatman et al. used Q-learning to make the robot follow the line \cite{saadatmand}. The authors trained the robot in a simulated environment and then conducted the test runs with a real robot. In their tests, the robot that used the simulation annealing-based Q-learning algorithm performed better than the typical proportional controller. In addition, this algorithm has the advantage that it does not need to be manually tuned, unlike the proportional controller.

Finally, Almeida et al. compared different control strategies for autonomous line-following robots \cite{almeida}. The authors compared a variety of PID, Fuzzy and Neural Network (NN) controllers and found that NN controllers perform the best and simple proportional controllers the worst. Although neural networks perform best, the authors note that training is time-consuming, and the approach is generally unsuitable for microcontrollers with low processing power. Furthermore, they concluded that PD and PID controllers are hard to tune compared to Fuzzy controllers, where the programmer can define the rules more intuitively.

\subsection{Maze-solving}

Line-following robots can be programmed to perform simple tasks, such as following a continuous line, and more complex tasks, such as solving a maze of lines. In these more complex scenarios, the additional challenges include detecting intersections, choosing the correct path, and navigating through the intersections. This section will focus on comparing different algorithms used for maze solving in other studies.

Mishra and Bande presented three algorithms for maze solving using robots \cite{mishra}. The first algorithm presented is wall following, where the robot follows the right or left wall to find the goal. The authors note that this algorithm only works for mazes with specific properties. 
Secondly, they analyse Dijkstra's algorithm, which works for every type of maze. They state that it always finds the shortest path, but it has the drawback that the robot needs to scan the entire labyrinth before calculating the shortest path, which takes a lot of time and energy.
Finally, they propose the Flood fill algorithm, which performs better than the other two in terms of the time the robot takes to complete the maze.

Another study compared non-graph-theory (NGT) algorithms (left-wall-follower and right-wall-follower) with graph-theory (GT) algorithms (DFS, BFS and Flood fill) \cite{sadik}. The authors simulated solving mazes with the different algorithms  and concluded that GT algorithms perform better than NGT algorithms. Regarding GT algorithms, they suggest using BFS over Flood fill for giant mazes as it requires less processing power while still performing adequately. If the maze to solve is small, the authors suggest using Flood fill as it finds the shortest path quicker than BFS.

\subsection{Current state of research}

Line-following robots are designed to navigate complex environments by following a line with high accuracy and reliability. Various robot designs were developed and studied by researchers, including different motor configurations, sensor configurations and navigation algorithms. While most research has examined robot configurations for line following without intersections, more research is needed on complex line following where intersections must be scanned and navigated. In particular, line-following robots with moving sensors have been studied very little. Therefore, this work addresses this research gap by comparing a robot with a fixed sensor with a robot with a 180-rotating colour sensor while solving a maze.
